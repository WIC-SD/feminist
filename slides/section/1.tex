\subsection{封建女性地位}
\begin{frame}{“三从四德”下缺失自由和地位卑微}
    \begin{block}{周南·桃夭}
        \begin{verse}
            桃之夭夭,灼灼其华。之子于归,宜其室家。 \\
            桃之夭夭,有蕡其实。之子于归,宜其家室。 \\
            桃之夭夭,其叶蓁蓁。之子于归,宜其家人。 \\
        \end{verse}
    \end{block}
    \begin{block}{}
        《诗经》中女性出现的场景多为娘家与婆家,无论是出嫁前还是出嫁后,女性都是二门不出大门不迈,要么娘家,要么婆家,没有第三选择。娘家是指已婚妇人的出生成人之所,是少女时的家,是女子为人妇前的住所。古代女子需遵守“三从四德”“女子无才便是德”的说法,娘家教导女子成为淑女,遵守妇道,维护娘家形象,婆家检验女子是否具有妇德,是女子“施展所学”的地方。不管是娘家还是婆家,女性的生活是被安排的状态,没有自由,没有空间,生活的地方便只有闺阁那一亩三分地。为人妻前,遵从父母之命媒妁之言,娘家束缚她的思想;为人妻后,婆家延续对女性的管教,禁锢她的自由。
    \end{block}
\end{frame}

\begin{frame}{女性在人格与生活上的依附性}
    \begin{block}{《京华烟云》}
        \small
        孙曼娘是一个典型的传统女子形象。她一生都为她的丈夫亚平活着,哪怕只是一个拥抱,她便认定了他是亚平的人,即使男人病重也毅然决然的嫁给了他,并在亚平死后,坚持着“生是曾家人,死是曾家鬼”的信念而守活寡,哪怕在其拥有了养子后,又将生活重心放在了儿子身上,几乎没有真正的为自己独立的人格活过,始终将自己的人格依附在男人、家庭、孩子身上,更别说是生活了,恐怕自己一个人也是活不下去的。赫拉克利特说:“一个人的性格就是他的命运。”孙曼娘的悲剧就在于受到传统意识观念洗礼而丧失自我的依附性人格。
    \end{block}
    \begin{block}{《白鹿原》}
        \small
        田小娥无疑也是命运悲惨的女性形象代表 ,她自小被卖给能做她爷爷的郭举人做小妾,为了摆脱这种水深火热的生活状态,她主动依附于黑娃,在黑娃出走之后,又将活下去的希望寄托于白孝文的身上。生活与命运对于田小娥是不公平的,但在寻找出路的过程中,她本能的选择了“男人”,这种对男性的依附性使得她从不相信在那样的时代能够靠自己的努力活下去,而事实也证明了,她最终还是没能找到一个护她一世周全的男人,最终悲惨死去。
    \end{block}
\end{frame}



\subsection{近代女权思想的出现}
\begin{frame}{社会经济的发展}
    \begin{block}{}
        \begin{itemize}
            \item 小农经济时期:男人和女人在体质和体力上确实存在自然差异,这就有了男尊女卑的反映,女性受世俗观念的影响更是被禁锢在了家中;
            \item 民族资本主义:企业里上出现工作强度不太大但需要很细致地进行的一类工作,适合女性工人来进行生产。
        \end{itemize}
    \end{block}
    \begin{block}{}
        五四时期女权意识的出现得益于民族资本主义经济快速发展,女性也和男性一样工作,经济地位提高,思想得到解放,经济的发展影响了社会观念,当时的一些观念开始慢慢转变,这就为早期女权意识的出现创造了条件。
    \end{block}
\end{frame}

\begin{frame}{西方“自然法权”和“天赋人权”思潮的输入}
    \begin{block}{}
        中国社会经过1840年的鸦片战争,受到了极大的震动,中国传统文化遭遇了“数千年来未有之变局”,西方社会的各种思想伴随着坚船利炮和传教士的活动开始在中国传播,西方的法文化也开始影响近代中国。有学者指出:“打破中国法文化封闭状态的急先锋,是近代来华的西方传教士”。从1840年以来,西方传教士开始大量涌入中国,1876年在中国的新教传教士有473人,1889年达到1296人,到1910年超过了5000人。这些传教士在中国设立了一些文化出版机构,从事文化传播活动。
    \end{block}
    \begin{block}{}
        最为知名的是1887年英国传教士韦廉臣在上海创立了广学会,传教士们称要“把中国人的思想开放起来”,认为“他们所带来的信息,不仅可以解决中国道德和精神方面的问题,还能解决政治和经济方面的问题。”传教士们的活动影响了上至皇帝,下至黎民的许多人。他们通过广学会的翻译出版工作,介绍了伏尔泰、卢梭、孟德斯鸠、狄德罗等人的学说以及法律改革思想,宣传了人权观念、平等观念、法制观念,这在中国要求进步、要求认识外部世界、向西方寻求救国真理的知识分子中间产生了很大的共鸣,起到了思想启蒙的作用,为中国的变法维新运动提供了理论先导。
    \end{block}
\end{frame}

\begin{frame}{马克思主义思想的传入}
    \begin{block}{}
        五四时期的一些学者将马克思主义带入了中国,随着马克思主义在中国的传播,其中有关女性的思想也在中国传播开来,在五四运动的推动下,这种思想迅速传播并被人们所接受。人们开始关注到女性这个群体。在这时期,出现了一系列报道女性解放运动、解读女性内心、为女性争取权益和要求改变当下女性生活现状的报纸刊物,以专栏的形式针对女性相关话题与社会大众共同探讨。女性解放与反封建传统文化、传统伦理被放到了一起讨论,女权思想也被具体化为平等地接受教育、个人的婚恋自由、独立的人格进行社会交往、经济独立等。马克思主义女性解放思想在同中国实际相结合的实践中得到了广泛的传播及接受。
    \end{block}
\end{frame}

\begin{frame}{先进知识分子的推动}
    \begin{block}{}
        2019年4月30日,习近平总书记在纪念五四运动100周年大会上指出:“五四运动前后,我国一批先进知识分子和革命青年,在追求真理中传播新思想新文化,勇于打破封建思想的桎梏。”
    \end{block}
    \begin{block}{}
        \begin{itemize}
            \item 郑观应:民受生于天。天赋之以能力,使之博硕丰大,以遂其生,于是有民权焉。民权者,君不能夺之臣,父不能夺之子,兄不能夺之弟,夫不能夺之妇;
            \item 康有为:大同之世,天下为公,无有阶级,一切平等;
            \item 李大钊同志率先将马克思主义带入中国,并把唯物史观用于女权思想,让马克思主义妇女理论在中国得到了传播并开始被接受;
            \item 陈独秀曾他的文章中对“三纲五常”这一封建观念大肆鞭挞,倡导尊重个人独立自主的人格,女性要从对男性的依附中脱离出来;
            \item 鲁迅通过他的作品揭露封建社会中男女不平等的现象,严肃批判了当时社会中男性霸权对女性产生的压迫和伤害;
        \end{itemize}
    \end{block}
\end{frame}



\subsection{女权思想的主要表现}
\begin{frame}{争取受教育权}
    \begin{block}{}
        在封建社会中,认为女子应该无才,女性不用学习文化知识,上学的都是男性,女性无权受教育,就算有部分女性能够接受教育,也是旧式教育。传统的旧式教育总是绕不开三纲五常、三从四德,一代又一代的根深蒂固。

        近代女子学堂的开办就在新文化运动中艰难地开创起来,但上层社会人民生活的一个重要表现与欧美国家接轨,他们对女性的生活、权利、能力有了肯定和接受,并且愿意让其接受更深层次的教育,这类上层社会知识女性在不断解放思想的过程中逐渐形成一个重要群体——名媛淑女。
    \end{block}
\end{frame}

\begin{frame}{争取婚姻自由权}
    \begin{block}{}
        如何争取婚姻自由也是在五四时期被热议的话题。在封建社会里,都是以男性为中心,女性是完全听从于丈夫的,妻子没有决断事物的权利,全凭丈夫做主,如果有违背丈夫的意愿,就可能会被丈夫抛弃。封建社会里男人还被允许有三妻四妾,这样一来女性的地位就更加低下了。女性在家庭中地位低下造就了其悲剧的命运。

        五四时期,先进知识分子受西方文化的影响,对西方的自由婚姻观念十分推崇,致力于推动女性婚姻自由。他们反对包办婚姻,向封建婚姻发起了挑战。女性需要有自主选择的权利,尤其是在婚姻方面,不能再以“父母之命、媒妁之言”来束缚女性,剥夺她们婚姻自由的权利。中国的女性想要彻底的逃脱封建礼教的迫害,首先在婚姻方面就必须要获得自主的权利,婚姻是自己的事情就应该由自己做主,恋爱自由、婚姻自由,离婚和再婚自由,摆脱封建社会的守节守寡等封建思想的约束,这样女性的权益才能够真正地得到保障。
    \end{block}
\end{frame}

\begin{frame}{争取政治参与权}
    \begin{block}{}
        女性参政成为女性解放的重要标志,新文化运动时期的女性争取参政的运动,开始注重将其作为女性承担对于国家的责任和实现自我社会价值的重要途径。女性参政,通过获得政治上的权利来解除种种不平等的待遇,恢复天赋的人权,而且女性参与政治事务管理也是尽国民应尽的义务,帮助男子来维持国家。这充分反映新文化运动时期女性人格意识在政治领域的进一步觉醒和成熟。
    \end{block}
\end{frame}
